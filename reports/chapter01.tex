\documentclass{article}
\usepackage[utf8]{inputenc}
\usepackage[spanish,es-tabla,es-nodecimaldot]{babel}
\usepackage{enumitem}
\usepackage{graphicx}
\graphicspath{ {reports/figures/} }
\usepackage{textcomp}
\usepackage[dvipsnames]{xcolor}
\usepackage{upquote,listings}
%\usepackage{lscape}
\usepackage{pdflscape}

\title{Instituto Politécnico Nacional \\ Centro de Investigación en Computación
\\
\vspace{1cm}
    Minería de datos, técnicas y conceptos. \\ Ejercicios del capítulo 1 }
\vspace{3cm}
\author{
    Ana Maritza Bello Yañez
}

\date{\today}

\begin{document}
\maketitle

\begin{enumerate}
\item ¿Qué es minería de datos?

\textcolor{NavyBlue}{
La minería de datos es la detección y extracción de patrones interesantes o
desconocidos de un conjunto de datos. Refiriendose a interesantes como aquellos
que reunen las siguientes características:}

\vfill
\begin{tabular}{|p{2.8cm}|p{7.5cm}|}
\hline
\textcolor{NavyBlue}{ \textbf{Válidos}.} & Para que los patrones sean válidos el conjunto de datos deben de
estar limpios, es decir, no contener errores que puedan afectar los análisis de
los datos.\\
\hline
\textcolor{NavyBlue}{ \textbf{No triviales}.} & Los patrones deben de ser interesantes, no obvios o que se
puedan observar sin la necesidad de hacer un análisis.\\
\hline
\textcolor{NavyBlue}{ \textbf{Implícitos}.} & El conocimiento generado a partir de los datos debe de
provenir del mismo conjunto de datos.\\
\hline
\textcolor{NavyBlue}{ \textbf{Novedosos o previamente desconocidos}. } & Va de la mano con que no deben de
ser triviales. Es decir, deben de generar conocimiento de patrones no notados
con anterioridad.\\
\hline
\textcolor{NavyBlue}{\textbf{Potencialmente útiles}.} & Es decir, que la información generada debe de
servir para la toma de decisiones. \\
\hline
\end{tabular}
\vfill


\textcolor{NavyBlue}{La minería de datos involucra todo un proceso para transformar las grandes
cantidades de datos en conocimiento organizado.}


- ¿Es otra exageración?

\textcolor{NavyBlue}{No. La minería de datos surge como respuesta a la necesidad
de analizar grandes volúmenes de datos que se generan en la actualidad. Puede
verse como el resultado inminente de la evolución de las tecnologíass de la
información.}

- ¿Es una simple transformación o aplicación de la tecnología desarrollada de
bases de datos, estadística, aprendizaje máquina o reconocimiento de patrones?

\textcolor{NavyBlue}{No. La minería de datos es una disciplina que involucra a
todas estas áreas de la computación (bases de datos, estadística, aprendizaje
máquina y reconocimiento de patrones) en un proceso de descubrimiento de
conocimiento.}

- Ya presentamos una aproximación de la minería de datos como resultado de la
evolución de las tecnologías de bases de datos. ¿Consideras que la minería de
datos también es el resultado de la evolución del aprendizaje máquina?, ¿Podrías
presentar tales opiniones basado en el progreso histórico de esta disciplina?
Haz lo mismo para el caso de la estadística y el reconocimiento de patrones.

- Describe los pasos de un proceso de minería de datos como un proceso de
descubrimiento de conocimiento.

\begin{enumerate}
\item \textcolor{RoyalBlue}{Limpieza de datos.} Remover datos inconsistentes o duplicados.
\item \textcolor{RoyalBlue}{Integración de datos.} Combinar múltiples fuentes de
datos.
\item \textcolor{RoyalBlue}{Selección de datos.} Seleccionar los datos
relevantes para el análisis.
\item \textcolor{RoyalBlue}{Transformación de datos.} Convertir los datos a un
formato apropiado para el análisis.
\item \textcolor{RoyalBlue}{Minería de datos.} Aplicar métodos de minería de
datos apropiados para el problema.
\item \textcolor{RoyalBlue}{Interpretación y evaluación de los patrones
encontrados.} Evaluar los patrones encontrados y determinar si son válidos.
\item \textcolor{RoyalBlue}{Presentación del conocimiento.} Representar el
conocimiento encontrado de una manera que sea útil para el usuario.
    
\end{enumerate}



\pagebreak
\item ¿Cuál es la diferencia de una bodega de datos de una base de datos?, ¿En
qué son similares?

\textcolor{NavyBlue}{Una base de datos es una colección de datos relacionados y
que son almacenados en un medio persistente. Los datos están crudos, es decir no
podemos ver información resumida, sino que cada línea de la base de datos
corresponde a un registro. }

\textcolor{NavyBlue}{Una bodega de datos es una colección de datos ya procesados
o transformados (resumidos) que son almacenados en un medio persistente. Una
línea de la bodega de datos corresponde a un resumen de los datos de la base de
datos. Estos datos están orientados a un dominio específico, son integrados y
ayudan a la toma de decisiones en la organización.}

Ambas son similares en que son colecciones de datos almacenados en un medio
persistente (disco duro o cinta magnética).

\pagebreak
\item Define cada una de las siguientes funcionalidades de la minería de datos.
Presenta un ejemplo de cada una de ellas usando un ejemplo de una base de datos
de la vida real con la que estés familiarizado.

\begin{itemize}
    \item Caracterización:
    \item Discriminación:
    \item Asociación y análisis de correlación:
    \item Clasificación:
    \item Regresión:
    \item Agrupamiento:
    \item Análisis de valores atípicos:
\end{itemize}

\pagebreak
\item Presenta un ejemplo dónde la minería de datos es crucial para el éxito de
un negocio. ¿Qué funcionalidades de la minería de datos necesita este negocio?
Piensa en los tipos de patrones que podrían ser minados. ¿Estos patrones pueden
ser generados alternativamente por procesamientos de consultas o por un simple
análisis estadístico?

\pagebreak
\item ¿Explica la diferencia y similitud entre \textbf{clasificación y
similitud}, \textbf{caracterización y clustering} y entre \textbf{clasificación
y regresión}?

\pagebreak
\item Basado en tus observaciones, describe otro posible tipo de conocimiento
que necesita ser descubierto por métodos de minería de datos pero no ha sido
listado en este capítulo.------

\pagebreak
\item Los \textit{valores atípicos} frecuentemente se consideran como ruido. Sin
embargo, la \textit{basura de una persona} podría ser el \textit{tesoro de
otra}. Por ejemplo, las excepciones las transacciones de tarjetas de crédito
pueden ayudarnos a detectar uso fraudulento de tarjetas de crédito. Usando este
ejemplo, propón otros dos métodos que pueden ser usados para detectar valores
atípicos y discute cual es más confiable.

\pagebreak
\item Describe los \textbf{tres retos de la minería de datos} que tienen que ver
con \textbf{metodologías de la minería de datos} y \textbf{asuntos de
interacción de usuario}.

\pagebreak
\item ¿Cuáles son los mayores retos de la minería en grandes bases de datos (por
ejemplo billones de tuplas) en comparación con las pequeñas cantidades de datos?

\pagebreak
\item Describe los mayores retos de investigación de la minería de datos en un
dominio específico de aplicación tal como análisis de datos de streams/sensores,
datos espacio-temporales o bioinformática.

\end{enumerate}

\end{document}